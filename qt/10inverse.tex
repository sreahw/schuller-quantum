
This section devoted to the development of all the notions and results necessary to understand and prove the spectral theorem, stated below. 

\bt[Spectral theorem]
For every self-adjoint operator $A\cl \mathcal{D}_A\to \mathcal{H}$ there is a unique projection-valued measure $\mathrm{P}_{\negmedspace A}\cl \sigma(\mathcal{O}_{\R})\to\mathcal{L}(\mathcal{H})$ such that
\bse
A = \int_{\R}\! i_{\R} \, \d \mathrm{P}_{\negmedspace A} \equiv \int_{\R}\! \lambda \, \mathrm{P}_{\negmedspace A}(\d \lambda),
\ese
where $i_{\R}\cl \R\hookrightarrow \C$ is the inclusion of $\R$ into $\C$.
\et

While useful in theory, existence results are often of limited use in practice since they usually only tell us that something exists, and not how to construct it. However, we should note here that the proof of the spectral theorem is, in fact, constructive in nature. Hence, given any self-adjoint operator $A$, we will be able to explicitly determine its associated projection-valued measure $\mathrm{P}_{\negmedspace A}$ along the following steps.
\begin{enumerate}[label=(\roman*)]
\item For each $\psi\in\mathcal{H}$, construct the real-valued Borel measure $\mu^A_{\psi}\cl \sigma(\mathcal{O}_{\R})\to \R$ given by
\bse
\mu^A_{\psi} ((-\infty,\lambda]):= \lim_{\delta\to 0^+} \lim_{\varepsilon\to 0^+} \int_{-\infty}^{\lambda+\delta}\!\d t  \Im\langle\psi | R_A(t+\mathrm{i}\varepsilon)\psi\rangle ,
\ese
where $R_A\cl\rho(A)\to \mathcal{L}(\mathcal{H})$ is the resolvent map of $A$. This is know as the \emph{Stieltjes inversion formula}. Note that while not every element in $\sigma(\mathcal{O}_{\R})$ is of the form $(-\infty,\lambda]$, such Borel measurable sets do generate the entire $\sigma(\mathcal{O}_{\R})$ via unions, intersections and set differences. Hence, the value of $\mu^A_{\psi}(\Omega)$ for $\Omega\in\sigma(\mathcal{O}_{\R})$ can be determined by applying the corresponding formulae for measures, namely $\sigma$-additivity, continuity from above and measure of set differences.
\item For all $\psi,\varphi\in\mathcal{H}$, define the complex-valued Borel measure $\mu^A_{\psi,\varphi}\cl\sigma(\mathcal{O}_{\R})\to\C$ by
\bse
\mu^A_{\psi,\varphi}(\Omega):=\tfrac{1}{4}(\mu^A_{\psi+\varphi}(\Omega)-\mu^A_{\psi-\varphi}(\Omega)+\mathrm{i}\mu^A_{\psi-\mathrm{i}\varphi}(\Omega)-\mathrm{i}\mu^A_{\psi+\mathrm{i}\varphi}(\Omega)).
\ese
\item Define the projection-valued measure $\mathrm{P}_{\negmedspace A}\cl \sigma(\mathcal{O}_{\R})\to\mathcal{L}(\mathcal{H})$ by requiring $\mathrm{P}_{\negmedspace A}(\Omega)$, for each $\Omega\in\sigma(\mathcal{O}_{\R})$, to be the unique map in $\mathcal{L}(\mathcal{H})$ satisfying
\bse
\forall\, \psi,\varphi\in\mathcal{H}:\ \langle\psi|\mathrm{P}_{\negmedspace A}(\Omega)\varphi\rangle = \int_{\R}\!\chi_{\Omega}\,\d\mu^A_{\psi,\varphi}.
\ese
\end{enumerate}

We will now make all the notions and constructions used herein precise. In fact, we will present the relevant definitions and results by taking the inverse route, starting with projection-valued measures and arriving at their associated self-adjoint operators, obtaining (and proving) what we will call the inverse spectral theorem.

\subsection{Projection-valued measures}

Projection-valued measures are, unsurprisingly, objects sharing characteristics of both measures and projection operators.

\bd
A map $\mathrm{P}\cl \sigma(\mathcal{O}_{\R})\to\mathcal{L}(\mathcal{H})$ is called a \emph{projection-valued measure}\index{projection-valued measure} if it satisfies the following properties.
\ben[label=(\roman*)]
\item $\forall\, \Omega \in \sigma(\mathcal{O}_{\R}) : \ \mathrm{P}(\Omega)^* = \mathrm{P}(\Omega) $
\item $\forall\, \Omega \in \sigma(\mathcal{O}_{\R}) : \ \mathrm{P}(\Omega)\circ \mathrm{P}(\Omega) = \mathrm{P}(\Omega) $
\item $\mathrm{P}(\R)=\id_{\mathcal{H}}$
\item For any pairwise disjoint sequence $\{\Omega_n\}_{n\in\N}$ in $\sigma(\mathcal{O}_{\R})$ and any $\psi\in\mathcal{H}$,
\bse
\sum_{n=0}^{\infty}\mathrm{P}(\Omega_n)\psi = \mathrm{P}\biggl(\bigcup_{\, n=0}^{\infty}\Omega_n\biggr)\psi.
\ese
\een
\ed

\bl
Let $\mathrm{P}\cl \sigma(\mathcal{O}_{\R})\to\mathcal{L}(\mathcal{H})$ be a projection-valued measure. Then, for any $\Omega,\Omega_1,\Omega_2\in\sigma(\mathcal{O}_{\R})$,
\ben[label=(\roman*)]
\item $\mathrm{P}(\varnothing)=0$, where by $0$ we mean $0\in\mathcal{L}(\mathcal{H})$
\item $\mathrm{P}(\R\setminus \Omega)={\id_{\mathcal{H}}}-\mathrm{P}(\Omega)$
\item $\mathrm{P}(\Omega_1\cup\Omega_2)=\mathrm{P}(\Omega_1)+\mathrm{P}(\Omega_2)-\mathrm{P}(\Omega_1\cap\Omega_2)$
\item $\mathrm{P}(\Omega_1\cap\Omega_2)=\mathrm{P}(\Omega_1)\circ\mathrm{P}(\Omega_2)$
\item if $\Omega_1\subseteq\Omega_2$, then $\ran(\mathrm{P}(\Omega_1))\subseteq \ran(\mathrm{P}(\Omega_2))$.
\een
\el












