
Hilbert spaces are a special type of a more general class of spaces known as Banach spaces. We are interested in Banach spaces not just for the sake generality, but also because they naturally appear in Hilbert space theory. For instance, the space of bounded linear maps on a Hilbert space is not itself a Hilbert space, but only a Banach space.

\subsection{Generalities on Banach spaces}

We begin with some basis notions from metric space theory.

\bd
A \emph{metric space}\index{metric space} is a pair $(X,d)$, where $X$ is a set and $d$ is a \emph{metric} on $X$, that is, a map $d\cl X\times X \to \R$ satisfying
\ben[label=(\roman*)]
\item $d(x,x)\geq 0$ \hfill (non-negativity)
\item $d(x,y) = 0\ \Leftrightarrow \ x=y$ \hfill (identity of indiscernibles)
\item $d(x,y)=d(y,x)$ \hfill (symmetry)
\item $d(x,z)\leq d(x,y)+d(y,z)$ \hfill (triangle inequality)
\een
for all $x,y,z\in X$.
\ed

\bd
A sequence $\{x_n\}_{n\in \N}$ in a metric space $(X,d)$ is said to \emph{converge}\index{convergence} to an element $x\in X$, written $\displaystyle \lim_{n\to \infty}x_n=x$, if
\bse
\forall \, \varepsilon > 0 : \exists \, N \in \N : \forall \, n \geq N : \ d(x_n,x)<\varepsilon.
\ese
\ed

A sequence in a metric space can converge to at most one element.

\bd
A \emph{Cauchy sequence}\index{Cauchy sequence} in a metric space $(X,d)$ is a sequence $\{x_n\}_{n\in \N}$ such that
\bse
\forall \, \varepsilon > 0 : \exists \, N \in \N : \forall \, n,m \geq N : \ d(x_n,x_m)<\varepsilon.
\ese
\ed
Any convergent sequence is clearly a Cauchy sequence.

\vbox{\bd
A metric space $(X,d)$ is said to be \emph{complete}\index{completeness} if every Cauchy sequence converges to some $x\in X$.
\ed
A natural metric on a vector space is that induced by a norm.
\bd
A \emph{normed space}\index{normed space} is a (complex) vector space $(V,+,\cdot)$ equipped with a \emph{norm}, that is, a map $\|\cdot\|\cl V \to \R$ satisfying
\ben[label=(\roman*)]
\item $\|f\|\geq 0$ \hfill (non-negativity)
\item $\|f\| = 0 \ \Leftrightarrow\ f=0$ \hfill (definiteness)
\item $\|z\cdot f\|=|z|\|f\|$ \hfill (homogeneity/scalability)
\item $\|f+g\|\leq \|f\|+\|g\|$ \hfill (triangle inequality/sub-additivity)
\een
for all $f,g\in V$ and all $z\in \C$.
\ed}
One we have a norm $\|\cdot\|$ on $V$, we can define a metric $d$ on $V$ by 
\bse
d(f,g) := \|f-g\|.
\ese
Then we say that the normed space $(V,\|\cdot\|)$ is \emph{complete} if the metric space $(V,d)$, where $d$ is the metric induced by $\|\cdot\|$, is complete. Note that we will usually suppress inessential information in the notation, for example writing $(V,\|\cdot\|)$ instead of $(V,+,\cdot,\|\cdot\|)$.

\bd
A \emph{Banach space}\index{Banach space} is a complete normed vector space.
\ed

\be
The space $C^0_\C[0,1]:=\{f\cl [0,1]\to \C \mid f \text{ is continuous}\}$, where the continuity is with respect to the standard topologies on $[0,1]\subset \R$ and $\C$, is a Banach space. Let us show this in some detail.
\bq
\ben[label=(\alph*)]
\item First, define two operations $+,\cdot$ pointwise, that is, for any $x\in [0,1]$
\bse
(f+g)(x) := f(x)+g(x)\qquad \quad (z\cdot f)(x):=zf(x).
\ese
Suppose that $f,g\in C^0_\C[0,1]$, that is
\bse
\forall \, x_0\in [0,1]: \forall \, \varepsilon > 0 : \exists \, \delta > 0 : \forall \, x\in (x_0-\delta,x_0+\delta) : \ |f(x)-f(x_0)|<\varepsilon
\ese
and similarly for $g$. Fix $x_0\in[0,1]$ and $\varepsilon >0$. Then, there exist $\delta_1,\delta_2>0$ such that
\bi{c}
\forall \, x\in (x_0-\delta_1,x_0+\delta_1) : \ |f(x)-f(x_0)|<\tfrac{\varepsilon}{2}\phantom{.}\\
\forall \, x\in (x_0-\delta_2,x_0+\delta_2) : \ |g(x)-g(x_0)|<\tfrac{\varepsilon}{2}.
\ei
Let $\delta:=\min\{\delta_1,\delta_2\}$. Then, for all $x\in (x_0-\delta,x_0+\delta)$, we have
\bi{rCl}
|(f+g)(x)-(f+g)(x_0)| & := & |f(x)+g(x)-(f(x_0)+g(x_0))|\\
 & = & |f(x)-f(x_0)+g(x)-g(x_0))|\\
 & \leq & |f(x)-f(x_0)|+|g(x)-g(x_0))|\\
& < & \tfrac{\varepsilon}{2}+\tfrac{\varepsilon}{2}\\
& = & \varepsilon.
\ei
Since $x_0\in[0,1]$ was arbitrary, we have $f+g\in C^0_\C[0,1]$. Similarly, for any $z\in \C$ and $f\in C^0_\C[0,1]$, we also have $z\cdot f\in C^0_\C[0,1]$. It is immediate to check that the complex vector space structure of $\C$ implies that the operations
\bi{rrClcrrCl}
+\cl & C^0_\C[0,1] \times C^0_\C[0,1] & \to & C^0_\C[0,1] &\quad \qquad & \cdot \cl &\C \times C^0_\C[0,1] & \to & C^0_\C[0,1]\\
& (f,g) &\mapsto & f+g && & (z,f) & \mapsto & z\cdot f
\ei
make $(C^0_\C[0,1],+,\cdot)$ into a complex vector space.

\item Since $[0,1]$ is closed and bounded, it is compact and hence every complex-valued continuous function $f\cl [0,1]\to \C$ is bounded, in the sense that
\bse
\sup_{x\in[0,1]}|f(x)| < \infty.
\ese
We can thus define a norm on $C^0_\C[0,1]$, called the \emph{supremum} (or \emph{infinity}) \emph{norm}, by
\bse
\|f\|_{\infty} := \sup_{x\in[0,1]}|f(x)| .
\ese
Let us show that this is indeed a norm on $(C^0_\C[0,1],+,\cdot)$ by checking that the four defining properties hold. Let $f,g\in C^0_\C[0,1]$ and $z\in \C$. Then
\ben[label=(b.\roman*)]
\item $\displaystyle \|f\|_{\infty}:= \sup_{x\in[0,1]}|f(x)| \geq 0$ since $|f(x)|\geq 0$ for all $x\in [0,1]$.
\item $\displaystyle \|f\|_{\infty}=0\ \Leftrightarrow \sup_{x\in[0,1]}|f(x)| = 0$. By definition of supremum, we have
\bse
\forall \, x \in [0,1] : \ |f(x)|\leq \sup_{x\in[0,1]}|f(x)| = 0. 
\ese
But since we also have $|f(x)|\geq 0$ for all $x\in [0,1]$, $f$ is identically zero.
\item $\displaystyle \|z\cdot f\|_{\infty} := \sup_{x\in[0,1]}|zf(x)| = \sup_{x\in[0,1]}|z||f(x)|=|z|\sup_{x\in[0,1]}|f(x)| = |z|\|f\|_{\infty}$.
\item By using the triangle inequality for the modulus of complex numbers, we have
\bi{rCl}
\|f+g\|_{\infty} &:=& \sup_{x\in[0,1]}|(f+g)(x)|\\
&=& \sup_{x\in[0,1]}|f(x)+g(x)|\\
&\leq & \sup_{x\in[0,1]}(|f(x)|+|g(x)|)\\
& = & \sup_{x\in[0,1]}|f(x)|\,+\sup_{x\in[0,1]}|g(x)|\\
& = & \|f\|_{\infty}+\|g\|_{\infty}.
\ei
Hence, $(C^0_\C[0,1],\|\cdot\|_{\infty})$ is indeed a normed space.
\een
\item We now show that $C^0_\C[0,1]$ is complete. Let $\{f_n\}_{n\in \N}$ be a Cauchy sequence of functions in $C^0_\C[0,1]$, that is
\bse
\forall \, \varepsilon > 0 : \exists \, N \in \N : \forall \, n,m \geq N : \ \|f_n-f_m\|_{\infty} <\varepsilon.
\ese
We seek an $f\in C^0_\C[0,1]$ such that $\displaystyle \lim_{n\to\infty}f_n=f$. We will proceed in three steps.
\ben[label=(c.\roman*)]
\item Fix $y\in [0,1]$ and $\varepsilon >0$. By definition of supremum, we have
\bse
f_n(y) -f_m(y) \leq \sup_{x\in[0,1]}|f_n(x)-f_m(x)| =: \|f_n-f_m\|_{\infty}.
\ese
Hence, there exists $N\in \N$ such that
\bse
\forall \, n,m \geq N : \ f_n(y) -f_m(y)<\varepsilon,
\ese
that is, the sequence of complex numbers $\{f_n(y)\}_{n\in \N}$ is a Cauchy sequence. Since $\C$ is a complete metric space\footnote{The standard metric on $\C$ is induced by the modulus of complex numbers.}, there exists $z_y\in\C$ such that $\displaystyle \lim_{n\to \infty}f_n(y)=z_y$. 

Thus, we can define a function
\bi{rrCl}
f\cl & [0,1] & \to & \C\\
& x & \mapsto & z_x,
\ei
called the \emph{pointwise limit} of $f$, which by definition satisfies
\bse
\forall \, x \in [0,1] : \ \lim_{n\to \infty}f_n(x)=f(x).
\ese
Note that this does \emph{not} automatically imply that $\displaystyle \lim_{n\to \infty}f_n=f$ (converge with respect to the supremum norm), nor that $f\in C^0_\C[0,1]$, and hence we need to check separately that these do, in fact, hold.
\item First, let us check that $f\in C^0_\C[0,1]$, that is, $f$ is continuous. Let $x_0\in [0,1]$ and $\varepsilon>0$. For each $x\in [0,1]$, we have
\bi{rCl}
|f(x)-f(x_0)| & = & |f(x)-f_n(x)+f_n(x)-f_n(x_0)+f_n(x_0)-f(x_0)|\\
 & \leq & |f(x)-f_n(x)|+|f_n(x)-f_n(x_0)|+|f_n(x_0)-f(x_0)|.
\ei
Since $f$ is the pointwise limit of $\{f_n\}_{n\in\N}$, for each $x\in[0,1]$ there exists $N\in \N$ such that
\bse
\forall \, n \geq N : \ |f(x)-f_n(x)|<\tfrac{\varepsilon}{3}.
\ese
In particular, we also have
\bse
\forall \, n \geq N : \ |f_n(x_0)-f(x_0)|<\tfrac{\varepsilon}{3}.
\ese
Moreover, since $f_n\in C^0_\C[0,1]$ by assumption, there exists $\delta>0$ such that
\bse
\forall \, x\in (x_0-\delta,x_0+\delta)  : \ |f_n(x)-f_n(x_0)|<\tfrac{\varepsilon}{3}.
\ese
Fix $n\geq N$. Then, it follows that for all $x\in (x_0-\delta,x_0+\delta)$, we have
\bi{rCl}
|f(x)-f(x_0)|  & \leq & |f(x)-f_n(x)|+|f_n(x)-f_n(x_0)|+|f_n(x_0)-f(x_0)|\\
& < & \tfrac{\varepsilon}{3}+\tfrac{\varepsilon}{3}+\tfrac{\varepsilon}{3}\\
& = & \varepsilon.
\ei
Since $x_0\in[0,1]$ was arbitrary, we have $f\in C^0_\C[0,1]$.
\item Finally, it remains to show that $\displaystyle \lim_{n\to \infty}f_n=f$. To that end, let $\varepsilon>0$. By the triangle inequality for $\|\cdot\|_{\infty}$, we have
\bi{rCl}
\|f_n-f\|_{\infty} &=& \|f_n-f_m+f_m-f\|_{\infty}\\
&\leq& \|f_n-f_m\|_{\infty}+\|f_m-f\|_{\infty}.
\ei
Since $\{f_n\}_{n\in\N}$ is Cauchy by assumption, there exists $N_1\in \N$ such that
\bse
\forall \, n,m\geq N : \ \|f_n-f_m\|_{\infty} < \tfrac{\varepsilon}{2}.
\ese
Moreover, since $f$ is the pointwise limit of $\{f_n\}_{n\in\N}$, for each $x\in[0,1]$ there exists $N_2\in \N$ such that
\bse
\forall \, m \geq N_2 : \ |f_m(x)-f(x)|<\tfrac{\varepsilon}{2}.
\ese
By definition of supremum, we have
\bse
\forall \, m \geq N_2 : \ \|f_m-f\|_{\infty}= \sup_{x\in[0,1]}|f_m(x)-f(x)|\leq\tfrac{\varepsilon}{2}.
\ese
Let $N:=\max\{N_1,N_2\}$ and fix $m\geq N$. Then, for all $n\geq N$, we have
\bi{c}
\|f_n-f\|_{\infty} \leq \|f_n-f_m\|_{\infty}+\|f_m-f\|_{\infty} < \tfrac{\varepsilon}{2} + \tfrac{\varepsilon}{2} = \varepsilon.
\ei
Thus, $\displaystyle \lim_{n\to \infty}f_n = f$ and we call $f$ the \emph{uniform limit} of $\{f_n\}_{n\in\N}$.
\een
\een
This completes the proof that $(C^0_\C[0,1],\|\cdot\|_{\infty})$ is a Banach space.
\eq
\ee

\br
The previous example shows that checking that something is a Banach space, and the completeness property in particular, can be quite tedious. However, in the following, we will typically already be working with a Banach (or Hilbert) space and hence, rather than having to check that the completeness property holds, we will instead be able to use it to infer the existence (within that space) of the limit of any Cauchy sequence.
\er

\subsection{Bounded linear operators}

As usual in mathematics, once we introduce a new types of structure, we also want study maps between instances of those structures, with extra emphasis placed on the structure-preserving maps. We begin with linear maps from a normed space to a Banach space.

\bd
Let $(V,\|\cdot\|_V)$ be a normed space and $(W,\|\cdot\|_W)$ a Banach space. A linear map, also called a linear operator, $A\cl V\to W$ is said to be \emph{bounded} if
\bse
\sup_{f\in V}\frac{\|Af\|_W}{\|f\|_V} < \infty.
\ese
\ed
Note that the quotient is not defined for $f=0$. Hence, to be precise, we should write $V\setminus\{0\}$ instead of just $V$. Let us agree that is what mean in the above definition. There are several equivalent characterisations of the boundedness property.

\vbox{
\bp
\label{prp:boundequiv}
A linear operator $A\cl V\to W$ is bounded if, and only if, any of the following conditions are satisfied.
\ben[label=(\roman*)]
\item $\displaystyle \sup_{\|f\|_V=1}\|Af\|_W< \infty$
\item $\exists \, k > 0 : \forall \, f\in V: \ \|f\|_V \leq 1\, \Rightarrow \, \|Af\|_W \leq k$
\item $\exists \, k > 0 : \forall \, f\in V: \ \|Af\|_W\leq  k \|f\|_V$
\item the map $A\cl V \to W$ is continuous with respect to the topologies induced by the respective norms on $V$ and $W$
\item the map $A$ is continuous at $0\in V$. 
\een
\ep}

The first one of these follows immediately from the homogeneity of the norm. Indeed, suppose that $\|f\|_V\neq 1$. Then
\bse
\frac{\|Af\|_W}{\|f\|_V} = \|f\|_V^{-1}\|Af\|_W =\|A(\|f\|_V^{-1} f)\|_W = \|A\widetilde f\|_W 
\ese
where $\widetilde f:= \|f\|_V^{-1}f$ is such that $\|\widetilde f\|_V=1$. Hence, the boundedness property is equivalent to condition (i) above. 

\bd
Let $A\cl V \to W$ be a bounded operator. The \emph{operator norm}\index{operator norm} of $A$ is defined as
\bse
\|A\|:=\sup_{\|f\|_V=1} \|Af\|_W = \sup_{f\in V}\frac{\|Af\|_W}{\|f\|_V} .
\ese
\ed

\be
Let $\id_W\cl W\to W$ be the identity operator on a Banach space $W$. Then
\bse
\sup_{f\in W}\frac{\|\id_Wf\|_W}{\|f\|_W} =\sup_{f\in W}1 = 1 <\infty.
\ese
Hence, $\id_W$ is a bounded operator and has unit norm.
\ee

\be
Denote by $C^1_{\C}[0,1]$ the complex vector space of once continuously differentiable complex-valued functions on $[0,1]$. Since differentiability implies continuity, this is a subset of $C^0_{\C}[0,1]$. Moreover, since sums and scaling by a complex number of continuously differentiable functions are again continuously differentiable, this is, in fact, a vector subspace of $C^0_{\C}[0,1]$, and hence also a normed space with the supremum norm $\|\cdot\|_{\infty}$.

Consider the first derivative operator
\bi{rrCl}
D \cl & C^1_{\C}[0,1] & \to & C^0_{\C}[0,1]\\
& f & \mapsto & f'.
\ei
We know from undergraduate real analysis that $D$ is a linear operator. We will now show that $D$ is an unbounded\footnote{Some people take the term unbounded to mean ``not necessarily bounded''. We take it to mean ``definitely not bounded'' instead.} linear operator. That is,
\bse
\sup_{f\in C^1_{\C}[0,1]} \frac{\|Df\|_{\infty}}{\|f\|_{\infty}} = \infty.
\ese
Note that, since the norm is a function into the real numbers, both $\|Df\|_{\infty}$ and $\|f\|_{\infty}$ are always finite for any $f\in C^1_{\C}[0,1]$. Recall that the supremum of a set of real numbers is its least upper bound and, in particular, it need not be an element of the set itself. What we have to show is that the set
\bse
\biggl\{ \frac{\|Df\|_{\infty}}{\|f\|_{\infty}} \ \Big| \ f\in C^1_{\C}[0,1] \biggr\} \subset \R
\ese
contains arbitrarily large elements. One way to do this is to exhibit a positively divergent (or unbounded from above) sequence within the set. 

Consider the sequence $\{f_n\}_{n\geq 1}$ where $f_n(x):=\sin(2\pi nx)$. We know that sine is continuously differentiable, hence $f_n\in C^1_{\C}[0,1]$ for each $n\geq 1$, with
\bse
Df_n(x) = D(\sin(2\pi nx)) = 2\pi n\cos(2\pi nx).
\ese
We have
\bse
\|f_n\|_{\infty} = \sup_{x\in[0,1]}|f_n(x)| = \sup_{x\in[0,1]}|\sin(2\pi nx)| = \sup \, [-1,1] = 1 
\ese
and
\bse
\|Df_n\|_{\infty} = \sup_{x\in[0,1]}|Df_n(x)| = \sup_{x\in[0,1]}|2\pi n\cos(2\pi nx)| = \sup\, [-2\pi n,2\pi n] = 2\pi n.
\ese
Hence, we have
\bse
\sup_{f\in C^1_{\C}[0,1]} \frac{\|Df\|_{\infty}}{\|f\|_{\infty}} \geq \sup_{\{f_n\}_{n\geq 1}} \frac{\|Df\|_{\infty}}{\|f\|_{\infty}} = \sup_{n\geq 1}\,2\pi n = \infty,
\ese
which is what we wanted. As an aside, we note that $C^1_{\C}[0,1]$ is not complete with respect to the supremum norm, but it is complete with respect to the norm
\bse
\|f\|_{C^1}:=\|f\|_{\infty}+\|f'\|_{\infty}.
\ese
While the derivative operator is still unbounded with respect to this new norm, in general, the boundedness of a linear operator does depend on the choice of norms on its domain and target, as does the numerical value of the operator norm.
\ee

\br
Apart from the ``minor'' detail that in quantum mechanics we deal with Hilbert spaces, use a different norm than the supremum norm and that the (one-dimensional) momentum operator acts as $P(\psi):=-\mathrm{i}\hbar\psi'$, the previous example is a harbinger of the fact that the momentum operator in quantum mechanics is unbounded. This will be the case for the position operator $Q$ as well.
\er
\bl
\label{lem:forcompl}
Let $(V,\|\cdot\|)$ be a normed space. Then, addition, scalar multiplication, and the norm are all sequentially continuous. That is, for any sequences $\{f_n\}_{n\in \N}$ and $\{g_n\}_{n\in \N}$ in $V$ converging to $f\in V$ and $g\in V$ respectively, and any sequence $\{z_n\}_{n\in \N}$ in $\C$ converging to $z\in \C$, we have
\ben[label=(\roman*)]
\item $\displaystyle \lim_{n\to \infty}(f_n+g_n)=f+g$
\item $\displaystyle \lim_{n\to \infty}z_nf_n=zf$.
\item $\displaystyle \lim_{n\to \infty}\|f_n\|=\|f\|$
\een
\el

\bq
\ben[label=(\roman*)]
\item Let $\varepsilon >0$. Since $\displaystyle \lim_{n\to \infty}f_n=f$ and $\displaystyle \lim_{n\to \infty}g_n=g$ by assumption, there exist $N_1,N_2\in \N$ such that
\bi{c}
\forall \, n\geq N_1 : \ \|f-f_n\|<\tfrac{\varepsilon}{2}\phantom{.}\\
\forall \, n\geq N_2 : \ \|g-g_n\|<\tfrac{\varepsilon}{2}.
\ei
Let $N:=\max\{N_1,N_2\}$. Then, for all $n\geq N$, we have
\bi{rCl}
\|(f_n+g_n)-(f+g)\| &=& \|f_n-f+g_n-g\|\\
&\leq& \|f_n-f\|+\|g_n-g\|\\
& < & \tfrac{\varepsilon}{2}+\tfrac{\varepsilon}{2}\\
& = & \varepsilon.
\ei
Hence $\displaystyle \lim_{n\to \infty}(f_n+g_n)=f+g$.


\item Since $\{z_n\}_{n\in \N}$ is a convergent sequence in $\C$, it is bounded. That is,
\bse
\exists \, k>0 : \forall \, n\in \N : \ |z_n|\leq k.
\ese
Let $\varepsilon >0$. Since $\displaystyle \lim_{n\to \infty}f_n=f$ and $\displaystyle \lim_{n\to \infty}z_n=z$, there exist $N_1,N_2\in \N$ such that
\bi{l}
\forall \, n\geq N_1 : \ \|f-f_n\|<\frac{\varepsilon}{2k}\\
\forall \, n\geq N_2 : \ \|z-z_n\|<\frac{\varepsilon}{2\|f\|}.
\ei
Let $N:=\max\{N_1,N_2\}$. Then, for all $n\geq N$, we have
\bi{rCl}
\|z_nf_n-zf\| &=& \|z_nf_n-z_nf+z_nf-zf\|\\
&=& \|z_n(f_n-f)+(z_n-z)f\|\\
&\leq& \|z_n(f_n-f)\|+\|(z_n-z)f\|\\
&=& |z_n|\|f_n-f\|+|z_n-z|\|f\|\\
& < & k\frac{\varepsilon}{2k}+\frac{\varepsilon}{2\|f\|}\|f\|\\
& = & \varepsilon.
\ei
Hence $\displaystyle \lim_{n\to \infty}z_nf_n=zf$.
\item Let $\varepsilon >0$. Since  $\displaystyle \lim_{n\to \infty}f_n=f$, there exists $N\in \N$ such that
\bse
\forall \, n \geq N : \ \|f_n-f\|<\varepsilon.
\ese
By the triangle inequality, we have
\bse
\|f_n\| = \|f_n-f+f\| \leq \|f_n-f\|+\|f\|  
\ese
so that $\|f_n\|-\|f\|\leq \|f_n-f\|$. Similarly, $\|f\|-\|f_n\|\leq \|f-f_n\|$. Since
\bse
\|f-f_n\| = \|-(f_n-f)\| = |-1|\|f_n-f\| = \|f_n-f\|,
\ese
we have $-\|f_n-f\|\leq\|f_n\|-\|f\|\leq\|f_n-f\|$ or, by using the modulus,
\bse
\bigl| \|f_n\|-\|f\|\bigr| \leq \|f_n-f\|.
\ese
Hence, for all $n\geq N$, we have $\bigl| \|f_n\|-\|f\|\bigr| <\varepsilon$ and thus  $\displaystyle \lim_{n\to \infty}\|f_n\|=\|f\|$.
\qedhere
\een
\eq
Note that by taking $\{z_n\}_{n\in\N}$ to be the constant sequence whose terms are all equal to some fixed $z\in \C$, we have $\displaystyle \lim_{n\to \infty}zf_n=zf$ as a special case of (ii).

This lemma will take care of some of the technicalities involved in proving the following crucially important result.

\begin{theorem}
The set $\mathcal{L}(V,W)$ of bounded linear operators from a normed space $(V,\|\cdot\|_V)$ to a Banach space $(W,\|\cdot\|_W)$, equipped with pointwise addition and scalar multiplication and the operator norm, is a Banach space.
\end{theorem}

\bq
\ben[label=(\alph*)]
\item Define addition and scalar multiplication on $\mathcal{L}(V,W)$ by
\bse
(A+B)f := Af+Bf \quad \qquad (zA)f := zAf.
\ese
It is clear that both $A+B$ and $zA$ are linear operators.  Moreover, we have
\bi{rCl}
\sup_{f\in V}\frac{\|(A+B)f\|_W}{\|f\|_V} &:=& \sup_{f\in V}\frac{\|Af+Bf\|_W}{\|f\|_V} \\
 &\leq & \sup_{f\in V}\frac{\|Af\|_W+\|Bf\|_W}{\|f\|_V} \\
 &=& \sup_{f\in V}\frac{\|Af\|_W}{\|f\|_V} + \sup_{f\in V}\frac{\|Bf\|_W}{\|f\|_V} \\
 &<& \infty
\ei
since $A$ and $B$ are bounded. Hence, $A+B$ is also bounded and we have
\bse
\|A+B\|\leq \|A\|+\|B\|.
\ese
Similarly, for $zA$ we have 
\bi{rCl}
\sup_{f\in V}\frac{\|(zA)f\|_W}{\|f\|_V} &:=& \sup_{f\in V}\frac{\|zAf\|_W}{\|f\|_V} \\
 &= & \sup_{f\in V}\frac{|z|\|Af\|_W}{\|f\|_V} \\
 &=& |z|\sup_{f\in V}\frac{\|Af\|_W}{\|f\|_V} \\
 &<& \infty
\ei
since $A$ is bounded and $|z|$ is finite. Hence, $zA$ is bounded and we have
\bse
\|zA\|=|z|\|A\|.
\ese
Thus, we have two operations
\bi{rrClcrrCl}
+\cl & \mathcal{L}(V,W)\times \mathcal{L}(V,W) & \to & \mathcal{L}(V,W) &\qquad \quad & \cdot \cl & \C \times \mathcal{L}(V,W) & \to & \mathcal{L}(V,W)\\
& (A,B) & \mapsto & A+B && & (z,A) & \mapsto & zA
\ei
and it is immediate to check that the vector space structure of $W$ induces a vector space structure on $\mathcal{L}(V,W)$ with these operations.

\item We need to show that $(\mathcal{L}(V,W),\|\cdot\|)$ is a normed space, i.e.\ that $\|\cdot\|$ satisfies the properties of a norm. We have already shown two of these in part (a), namely
\ben[label=(b.\roman*),start=3]
\item $\|zA\|=|z|\|A\|$
\item $\|A+B\|\leq \|A\|+\|B\|$.
\een
The remaining two are easily checked.
\ben[label=(b.\roman*)]
\item $\displaystyle \|A\|:=\sup_{f\in V}\frac{\|Af\|_W}{\|f\|_V}\geq 0$ since $\|\cdot\|_V$ and $\|\cdot\|_W$ are norms.
\item Again, by using the fat that $\|\cdot\|_W$ is a norm,
\bi{rCl"s}
\|A\|=0 &\ \Leftrightarrow\ & \sup_{f\in V}\frac{\|Af\|_W}{\|f\|_V}= 0\\
& \Leftrightarrow & \forall \, f\in V:\ \|Af\|_W=0\\
& \Leftrightarrow & \forall \, f\in V:\ Af=0\\
& \Leftrightarrow & \ A=0.
\ei
\een
Hence, $(\mathcal{L}(V,W),\|\cdot\|)$ is a normed space.

\item The heart of the proof is showing that $(\mathcal{L}(V,W),\|\cdot\|)$ is complete. We will proceed in three steps, analogously to the case of $C_{\C}^0[0,1]$.
\ben[label=(c.\roman*)]
\item Let $\{A_n\}_{n\in \N}$ be a Cauchy sequence in $\mathcal{L}(V,W)$. Fix $f\in V$ and let $\varepsilon >0$. Then, there exists $N\in \N$ such that
\bse
\forall \, n,m\geq N : \ \|A_n-A_m\|<\frac{\varepsilon}{\|f\|_V}.
\ese
Then, for all $n,m\geq N$, we have
\bi{rCl}
\|A_nf-A_mf\|_W & = & \|(A_n-A_m)f\|_W\\
& = & \|f\|_V\frac{\|(A_n-A_m)f\|_W}{\|f\|_V}\\
& \leq & \|f\|_V\sup_{f\in V}\frac{\|(A_n-A_m)f\|_W}{\|f\|_V}\\
& =: & \|f\|_V\|A_n-A_m\|\\
& < & \|f\|_V \frac{\varepsilon}{\|f\|_V}\\
& = & \varepsilon.
\ei
(Note that if $f=0$, we simply have $\|A_nf-A_mf\|_W=0<\varepsilon$ and, in the future, we will not mention this case explicitly.) Hence, the sequence $\{A_nf\}_{n\in \N}$ is a Cauchy sequence in $W$. Since $W$ is a Banach space, the limit $\lim_{n\to \infty}A_nf$ exists and is an element of $W$. Thus, we can define the operator
\bi{rrCl}
A\cl & V & \to & W\\
& f & \mapsto & \lim_{n\to \infty}A_nf,
\ei
called the pointwise limit of $\{A_n\}_{n\in \N}$.
\item We now need to show that $A\in \mathcal{L}(V,W)$. This is where the previous lemma comes in handy. For linearity, let $f,g\in V$ and $z\in \C$. Then
\bi{rCl}
A(zf+g) &:=& \lim_{n\to \infty} A_n(zf+g)\\
& = & \lim_{n\to \infty} (zA_nf+A_ng)\\
& = & z\lim_{n\to \infty} A_nf+\lim_{n\to \infty}A_ng\\
&=:& zAf+Ag
\ei
where we have used the linearity of each $A_n$ and part (i) and (ii) of \Cref{lem:forcompl}.
For boundedness, part (ii) and (iii) of \Cref{lem:forcompl} yield
\bi{rCl}
\|Af\|_W  & = & \lim_{n\to \infty} \|A_nf\|_W\\
& = & \lim_{n\to \infty} \|f\|_V\frac{\|A_nf\|_W}{\|f\|_V}\\
& \leq & \lim_{n\to \infty} \|f\|_V\sup_{f\in V}\frac{\|A_nf\|_W}{\|f\|_V}\\
& = & \|f\|_V \lim_{n\to \infty} \|A_n\|
\ei
for any $f\in V$. By rearranging, we have
\bse
\forall\, f\in V:\ \frac{\|Af\|_W}{\|f\|_V} \leq  \lim_{n\to \infty} \|A_n\|.
\ese
Hence, to show that $A$ is bounded, it suffices to show that the limit on the right hand side is finite. Let $\varepsilon > 0$. Since $\{A_n\}_{n\in \N}$ is a Cauchy sequence, there exists $N\in \N$ such that
\bse
\forall \, n,m \geq N : \ \|A_n-A_m\|<\varepsilon.
\ese
Then, by the proof of part (i) of \Cref{lem:forcompl}, we have
\bse
\bigl| \|A_n\|-\|A_m\|\bigr| \leq  \|A_n-A_m\|<\varepsilon
\ese
for all $n,m \geq N$. Hence, the sequence of real numbers $\{\|A_n\|\}_{n\in \N}$ is a Cauchy sequence. Since $\R$ is complete, this sequence converges to some real number $r\in\R$. Therefore
\bse
\sup_{f\in V} \frac{\|Af\|_W}{\|f\|_V} \leq  \lim_{n\to \infty} \|A_n\| = r < \infty
\ese
and thus $A\in \mathcal{L}(V,W)$.

\item To conclude, we have to show that $\displaystyle \lim_{n\to\infty}A_n=A$. Let $\varepsilon >0$. Then
\bi{c}
\|A_n-A\| = \|A_n+A_m-A_m-A\| \leq \|A_n-A_m\|+ \|A_m-A\|.  
\ei
Since $\{A_n\}_{n\in \N}$ is Cauchy, there exists $N_1\in \N$ such that
\bse
\forall \, n,m\geq N_1 : \ \|A_n-A_m\| < \tfrac{\varepsilon}{2}.
\ese
Moreover, since $A$ is the pointwise limit of $\{A_n\}_{n\in \N}$, for any $f\in V$ there exists $N_2\in \N$ such that
\bse
\forall \, m\geq N_2 : \ \|A_mf-Af\|_W < \frac{\varepsilon\|f\|_V}{2}
\ese
and hence, for all $m\geq N_2$
\bse
\|A_m-A\| := \sup_{f\in V}\frac{\|A_mf-Af\|_W}{\|f\|_V} \leq \frac{\tfrac{\varepsilon\|f\|_V}{2}}{\|f\|_V} = \frac{\varepsilon}{2} 
\ese
Let $N:=\max\{N_1,N_2\}$ and fix $m\geq N$. Then, for all $n\geq N$, we have
\bse
\|A_n-A\| \leq \|A_n-A_m\|+ \|A_m-A\| <\tfrac{\varepsilon}{2}+\tfrac{\varepsilon}{2}=\varepsilon. 
\ese
Thus, $\displaystyle \lim_{n\to\infty}A_n=A$ and we call $A$ the uniform limit of $\{A_n\}_{n\in \N}$.
\een
\een
This concludes the proof that $(\mathcal{L}(V,W),\|\cdot\|)$ is a Banach space.
\eq

\br
Note that if $V$ and $W$ are normed spaces, then $\mathcal{L}(V,W)$ is again a normed space, while for $\mathcal{L}(V,W)$ to be a Banach space it suffices that $W$ be a Banach space.
\er

\br
In the proof that $\mathcal{L}(V,W)$ is a Banach space, we have shown a useful inequality which we restate here for further emphasis. If $A\cl V\to W$ is bounded, then 
\bse
\forall \, f\in V :\ \|Af\|_W\leq\|A\|\|f\|_V
\ese
\er

The following is an extremely important special case of $\mathcal{L}(V,W)$.

\bd
Let $V$ be a normed space. Then $V^*:=\mathcal{L}(V,\C)$ is called the \emph{dual}\index{dual} of $V$.
\ed

Note that, since $\C$ is a Banach space, the dual of a normed space is a Banach space. The elements of $V^*$ are variously called \emph{covectors} or \emph{functionals} on $V$. 
\br
You may recall from undergraduate linear algebra that the dual of a vector space was defined to be the vector space of \emph{all} linear maps $V\to\C$, rather than just the bounded ones. This is because, in finite dimensions, all linear maps are bounded. So the two definitions agree as long as we are in finite dimensions. If we used the same definition for the infinite-dimensional case, then $V^*$ would lack some very desirable properties, such as that of being a Banach space.
\er
The dual space can be used to define a weaker notion of convergence called, rather unimaginatively, weak convergence. 

\bd
A sequence $\{f_n\}_{n\in \N}$ is said to \emph{converge weakly}\index{weak convergence} to $f\in V$ if
\bse
\forall \, \varphi\in V^* : \lim_{n\to\infty}\varphi(f_n) = \varphi(f).
\ese
\ed

Note that $\{\varphi(f_n)\}_{n\in \N}$ is just a sequence of complex numbers. To indicate that the sequence $\{f_n\}_{n\in \N}$ converges weakly to $f\in V$ we write
\bse
\wlim_{n\to \infty} f_n = f.
\ese
In order to further emphasise the distinction with weak convergence, we may say that $\{f_n\}_{n\in \N}$ converges \emph{strongly} to $f\in V$ if it converges according to the usual definition, and we will write accordingly
\bse
\slim_{n\to \infty} f_n = f.
\ese

\bp
Let $\{f_n\}_{n\in \N}$ be a sequence in a normed space $(V,\|\cdot\|_V)$. If $\{f_n\}_{n\in \N}$ converges strongly to $f\in V$, then it also converges weakly to $f\in V$, i.e.\
\bse
\slim_{n\to \infty} f_n = f \quad \Rightarrow \quad \wlim_{n\to \infty} f_n = f.
\ese
\ep

\bq
Let $\varepsilon >0$ and let $\varphi\in V^*$. Since $\{f_n\}_{n\in \N}$ converges strongly to $f\in V$, there exists $N\in\N$ such that
\bse
\forall \, n\geq N : \ \|f_n-f\|_V<\frac{\varepsilon}{\|\varphi\|}.
\ese
Then, since $\varphi\in V^*$ is bounded, we have
\bi{rCl}
|\varphi (f_n)-\varphi (f)| & = & |\varphi ( f_n-f)|\\
& \leq & \|\varphi\|\|f_n-f\|_V\\
&<& \|\varphi\|\frac{\varepsilon}{\|\varphi\|}\\
& = & \varepsilon
\ei
for any $n\geq N$. Hence, $\displaystyle  \lim_{n\to \infty} \varphi(f_n) = \varphi(f)$. That is, $\displaystyle  \wlim_{n\to \infty} f_n = f$.
\eq

\subsection{Extension of bounded linear operators}

Note that, so far, we have only considered bounded linear maps $A\cl\mathcal{D}_A\to W$ where $\mathcal{D}_A$ is the whole of $V$, rather than a subspace thereof. The reason for this is that we will only consider densely defined linear maps in general, and any bounded linear map from a dense subspace of $V$ can be extended to a bounded linear map from the whole of $V$. Moreover, the extension is unique. This is the content of the so-called BLT\footnote{Bounded Linear Transformation, \emph{not} Bacon, Lettuce, Tomato. \begin{tikzpicture}[baseline={($ (current bounding box.center)- (0,-8pt) $)}]\includegraphics[scale=0.12]{graphics/blt}\end{tikzpicture}} theorem.

\bl
Let $(V,\|\cdot\|)$ be a normed space and let $\mathcal{D}_A$ be a dense subspace of $V$. Then, for any $f\in V$, there exists a sequence $\{\alpha_n\}_{n\in \N}$ in $\mathcal{D}_A$ which converges to $f$.
\el
\bq
Let $f\in V$. Clearly, there exists a sequence $\{f_n\}_{n\in \N}$ in $V$ which converges to $f$ (for instance, the constant sequence). Let $\varepsilon >0$. Then, there exists $N\in \N$ such that
\bse
\forall \, n\geq N : \ \|f_n-f\| < \tfrac{\varepsilon}{2}.
\ese
Since $\mathcal{D}_A$ is dense in $V$ and each $f_n\in V$, we have
\bse
\forall \, n\in \N : \exists\, \alpha_n\in\mathcal{D}_A : \ \|\alpha_n-f_n\|<\tfrac{\varepsilon}{2}.\\[8pt]
\ese
The sequence $\{\alpha_n\}_{n\in \N}$ is a sequence in $\mathcal{D}_A$ and we have
\bi{rCl}
\|\alpha_n-f\|&=&\|\alpha_n-f_n+f_n-f\|\\
&\leq &\|\alpha_n-f_n\|+\|f_n-f\| \\
&<&\tfrac{\varepsilon}{2}+\tfrac{\varepsilon}{2}\\
&=&\varepsilon
\ei
for all $n\geq N$. Hence $\displaystyle \lim_{n\to\infty}\alpha_n=f$.
\eq

\bd
Let $V,W$ be vector spaces and let $A\cl\mathcal{D}_A\to W$ be a linear map, where $\mathcal{D}_A\subseteq V$. An \emph{extension}\index{extension} of $A$ is a linear map $\widehat A\cl V\to W$ such that
\bse
\forall \, \alpha\in \mathcal{D}_A : \ \widehat A \alpha = A\alpha.
\ese
\ed

\bt[BLT theorem]
Let $V$ be a normed space and $W$ a Banach space. Any densely defined linear map $A\cl\mathcal{D}_A\to W$ has a unique extension $\widehat A\cl V\to W$ such that $\widehat A$ is bounded. Moreover, $\|{\widehat A}\|=\|A\|$.
\et

\bq
\ben[label=(\alph*)]
\item Let $A\in \mathcal{L}(\mathcal{D}_A,W)$. Since $\mathcal{D}_A$ is dense in $V$, for any $f\in V$ there exists a sequence $\{\alpha_n\}_{n\in \N}$ in $\mathcal{D}_A$ which converges to $f$. Moreover, since $A$ is bounded, we have 
\bse
\forall \, n\in \N : \ \|A\alpha_n-A\alpha_m\|_W \leq \|A\| \|\alpha_n-\alpha_m\|_V,
\ese
from which it quickly follows that $\{A\alpha_n\}_{n\in \N}$ is Cauchy in $W$. As $W$ is a Banach space, this sequence converges to an element of $W$ and thus we can define
\bi{rrCl}
\widehat A \cl & V &\to & W\\
& f & \mapsto & \lim_{n\to\infty}A\alpha_n,
\ei
where $\{\alpha_n\}_{n\in \N}$ is any sequence in $\mathcal{D}_A$ which converges to $f$.
\item First, let us show that $\widehat A$ is well-defined. Let $\{\alpha_n\}_{n\in \N}$ and $\{\beta_n\}_{n\in \N}$ be two sequences in $\mathcal{D}_A$ which converge to $f\in V$ and let $\varepsilon > 0$. Then, there exist $N_1,N_2\in \N$ such that
\bi{c}
\forall \, n\geq N_1 : \ \|\alpha_n-f\|_V < \frac{\varepsilon}{2\|A\|}\\
\forall \, n\geq N_2 : \ \|\beta_n-f\|_V < \frac{\varepsilon}{2\|A\|}.
\ei
Let $N:=\max\{N_1,N_2\}$. Then, for all $n\geq N$, we have
\bi{rCl}
\|A\alpha_n-A\beta_n\|_W & = & \|A(\alpha_n-\beta_n)\|_W\\
&\leq & \|A\| \| \alpha_n-\beta_n\|_V\\
& = & \|A\| \| \alpha_n-f+f-\beta_n\|_V\\
&\leq & \|A\| (\| \alpha_n-f\|_V+\|f-\beta_n\|_V)\\
&< & \|A\| \Bigl(\frac{\varepsilon}{2\|A\|}+\frac{\varepsilon}{2\|A\|}\Bigr)\\
&=&\varepsilon,
\ei
where we have used the fact that $A$ is bounded. Thus, we have shown
\bse
\lim_{n\to\infty}(A\alpha_n-A\beta_n) = 0.
\ese
Then, by using \Cref{lem:forcompl} and rearranging, we find
\bse
\lim_{n\to\infty}A\alpha_n = \lim_{n\to\infty}A\beta_n,
\ese
that is, $\widehat A$ is indeed well-defined.

\item To see that $\widehat A$ is an extension of $A$, let $\alpha\in\mathcal{D}_A$. The constant sequence $\{\alpha_n\}_{n\in \N}$ with $\alpha_n=\alpha$ for all $n\in \N$ is a sequence in $\mathcal{D}_A$ converging to $\alpha$. Hence
\bse
\widehat A \alpha := \lim_{n\to\infty} A\alpha_n =  \lim_{n\to\infty} A\alpha = A\alpha.
\ese

\item We now check that $A\in\mathcal{L}(V,W)$. For linearity, let $f,g\in V$ and $z\in \C$. As $\mathcal{D}_A$ is dense in $V$, there exist sequences $\{\alpha_n\}_{n\in \N}$ and $\{\beta_n\}_{n\in \N}$ in $\mathcal{D}_A$ converging to $f$ and $g$, respectively. Moreover, as $\mathcal{D}_A$ is a subspace of $V$, the sequence $\{\gamma_n\}_{n\in \N}$ given by
\bse
\gamma_n:=z\alpha_n+\beta_n
\ese
is again a sequence in $\mathcal{D}_A$ and, by \Cref{lem:forcompl}, 
\bse
\lim_{n\to\infty}\gamma_n = zf+g.
\ese
Then, we have
\bi{rCl}
\widehat A (zf+g) &:=& \lim_{n\to\infty} A\gamma_n \\
&=& \lim_{n\to\infty}A(z\alpha_n+\beta_n)\\
&=& \lim_{n\to\infty}(zA\alpha_n+A\beta_n)\\
&=& z\lim_{n\to\infty}A\alpha_n+\lim_{n\to\infty}A\beta_n\\
&=:& z\widehat A f+\widehat Ag .
\ei
For boundedness, let $f\in V$ and $\{\alpha_n\}_{n\in \N}$ a sequence in $\mathcal{D}_A$ which converges to $f$. Then, since $A$ is bounded,
\bi{rCl}
\|\widehat A f\|_W &:=& \bigl\|\lim_{n\to\infty}A\alpha_n\bigr\|_W\\
 &=& \lim_{n\to\infty}\|A\alpha_n\|_W\\
 & \leq &\lim_{n\to\infty}\|A\|\|\alpha_n\|_V\\
 & = &\|A\|\lim_{n\to\infty}\|\alpha_n\|_V\\
 & = &\|A\|\|f\|_V.
\ei
Therefore
\bse
\sup_{f\in V} \frac{\|\widehat A f\|_W}{\|f\|_V} \leq \sup_{f\in V} \frac{\|A\|\|f\|_V}{\|f\|_V} =\sup_{f\in V} \|A\| =\|A\| < \infty
\ese
and hence $\widehat A$ is bounded.
\item For uniqueness, suppose that $\widetilde A\in\mathcal{L}(V,W)$ is another extension of $A$. Let $f\in V$ and $\{\alpha_n\}_{n\in \N}$ a sequence in $\mathcal{D}_A$ which converges to $f$. Then, we have
\bse
\|\widetilde A f - A\alpha_n\|_W = \|\widetilde A f - \widetilde A\alpha_n\|_W \leq  \|\widetilde A\| \|f-\alpha_n\|_V .
\ese
It follows that
\bse
\lim_{n\to\infty}(\widetilde A f - A\alpha_n) = 0
\ese
and hence, for all $f\in V$,
\bse
\widetilde Af = \lim_{n\to\infty} A\alpha_n =: \widehat A f.
\ese
Therefore, $\widetilde A = \widehat A$.
\item Finally, we have already shown in part (d) that
\bse
\|\widehat A\|:=\sup_{f\in V} \frac{\|\widehat A f\|_W}{\|f\|_V} \leq \|A\|.
\ese
On the other hand, since $\mathcal{D}_A\subseteq V$, we must also have
\bse
\|A\|:=\sup_{f\in \mathcal{D}_A} \frac{\| A f\|_W}{\|f\|_V} = \sup_{f\in \mathcal{D}_A} \frac{\|\widehat A f\|_W}{\|f\|_V} \leq \sup_{f\in V} \frac{\|\widehat A f\|_W}{\|f\|_V}=: \|\widehat A\|.
\ese
Hence, we also have $\|A\|\leq\|\widehat A\|$. Thus, $\|\widehat A\|=\|A\|$.\qedhere
\een
\eq

\br
Note a slight abuse of notation in the equality $\|\widehat A\|=\|A\|$. The linear maps $\widehat A$ and $A$ belong to $\mathcal{L}(V,W)$ and $\mathcal{L}(\mathcal{D}_A,W)$, respectively. These are different normed (in fact, Banach) spaces and, in particular, carry \emph{different} norms. To be more precise, we should have written
\bse
\|\widehat A\|_{\mathcal{L}(V,W)} = \|A\|_{\mathcal{L}(\mathcal{D}_A,W)},
\ese
where 
\bse
\|\widehat A\|_{\mathcal{L}(V,W)} := \sup_{f\in V} \frac{\| \widehat A f\|_W}{\|f\|_V} \qquad \text{and}\qquad
\| A\|_{\mathcal{L}(\mathcal{D}_A,W)} := \sup_{f\in \mathcal{D}_A} \frac{\| A f\|_W}{\|f\|_V}.
\ese
\er





















